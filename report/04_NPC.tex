\section{NPC}
\label{sec:npc}

The NPC algorithm tries to construct the network exploiting the conditional independence in the samples.
It perform some statistical test for conditional independence for each pair of random variables.
An undirected edge is added between each pair of variables which are not statistically independent, then the conflicts (eg. cycles) are resolved using some heuristic or the user input.
Note that the algorithm is not able to decide the direction of causal relationships, since this is proven to be impossible.

The algorithm is executed with the default settings.
During the parameters learning with the \ac{EM} algorithm, the experience is set to once ``active'' and once ``inactive'' for all genes, even if they are never observed active or never observed inactive.
This forces the algorithm to never assign zeros probabilities to random variables, which would create problem to the inference process.
\cref{tab:npc} summarises the performances of the \num{5} networks learned using NPC.

\begin{table}
	\centering
	\caption{NPC network's performances}
	\label{tab:npc}
	\begin{tabular}{ccccc}
    	\toprule
    	    \multicolumn{1}{c}{k} &
    		\multicolumn{1}{c}{accuracy} &
    		\multicolumn{1}{c}{recall} &
    		\multicolumn{1}{c}{precision} &
    		\multicolumn{1}{c}{f1} \\
    	\midrule
    		1   & 0.80 & 1.00 & 0.40 & 0.57 \\
    		2   & 0.80 & 1.00 & 0.63 & 0.77 \\
    		3   & 0.79 & 1.00 & 0.63 & 0.77 \\
    		4   & 0.64 & 0.86 & 0.60 & 0.71 \\
    		5   & 0.57 & 0.83 & 0.50 & 0.63 \\[2pt]
    		\hline
    		avg & 0.72 & 0.92 & 0.56 & 0.70 \Tstrut\Bstrut\\
    	\bottomrule    
	\end{tabular}
\end{table}
